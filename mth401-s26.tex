% Options for packages loaded elsewhere
\PassOptionsToPackage{unicode}{hyperref}
\PassOptionsToPackage{hyphens}{url}
\PassOptionsToPackage{dvipsnames,svgnames,x11names}{xcolor}
%
\documentclass[
  letterpaper,
  DIV=11,
  numbers=noendperiod]{scrartcl}

\usepackage{amsmath,amssymb}
\usepackage{iftex}
\ifPDFTeX
  \usepackage[T1]{fontenc}
  \usepackage[utf8]{inputenc}
  \usepackage{textcomp} % provide euro and other symbols
\else % if luatex or xetex
  \usepackage{unicode-math}
  \defaultfontfeatures{Scale=MatchLowercase}
  \defaultfontfeatures[\rmfamily]{Ligatures=TeX,Scale=1}
\fi
\usepackage{lmodern}
\ifPDFTeX\else  
    % xetex/luatex font selection
\fi
% Use upquote if available, for straight quotes in verbatim environments
\IfFileExists{upquote.sty}{\usepackage{upquote}}{}
\IfFileExists{microtype.sty}{% use microtype if available
  \usepackage[]{microtype}
  \UseMicrotypeSet[protrusion]{basicmath} % disable protrusion for tt fonts
}{}
\makeatletter
\@ifundefined{KOMAClassName}{% if non-KOMA class
  \IfFileExists{parskip.sty}{%
    \usepackage{parskip}
  }{% else
    \setlength{\parindent}{0pt}
    \setlength{\parskip}{6pt plus 2pt minus 1pt}}
}{% if KOMA class
  \KOMAoptions{parskip=half}}
\makeatother
\usepackage{xcolor}
\setlength{\emergencystretch}{3em} % prevent overfull lines
\setcounter{secnumdepth}{2}
% Make \paragraph and \subparagraph free-standing
\ifx\paragraph\undefined\else
  \let\oldparagraph\paragraph
  \renewcommand{\paragraph}[1]{\oldparagraph{#1}\mbox{}}
\fi
\ifx\subparagraph\undefined\else
  \let\oldsubparagraph\subparagraph
  \renewcommand{\subparagraph}[1]{\oldsubparagraph{#1}\mbox{}}
\fi


\providecommand{\tightlist}{%
  \setlength{\itemsep}{0pt}\setlength{\parskip}{0pt}}\usepackage{longtable,booktabs,array}
\usepackage{calc} % for calculating minipage widths
% Correct order of tables after \paragraph or \subparagraph
\usepackage{etoolbox}
\makeatletter
\patchcmd\longtable{\par}{\if@noskipsec\mbox{}\fi\par}{}{}
\makeatother
% Allow footnotes in longtable head/foot
\IfFileExists{footnotehyper.sty}{\usepackage{footnotehyper}}{\usepackage{footnote}}
\makesavenoteenv{longtable}
\usepackage{graphicx}
\makeatletter
\def\maxwidth{\ifdim\Gin@nat@width>\linewidth\linewidth\else\Gin@nat@width\fi}
\def\maxheight{\ifdim\Gin@nat@height>\textheight\textheight\else\Gin@nat@height\fi}
\makeatother
% Scale images if necessary, so that they will not overflow the page
% margins by default, and it is still possible to overwrite the defaults
% using explicit options in \includegraphics[width, height, ...]{}
\setkeys{Gin}{width=\maxwidth,height=\maxheight,keepaspectratio}
% Set default figure placement to htbp
\makeatletter
\def\fps@figure{htbp}
\makeatother

\KOMAoption{captions}{tableheading}
\makeatletter
\@ifpackageloaded{tcolorbox}{}{\usepackage[skins,breakable]{tcolorbox}}
\@ifpackageloaded{fontawesome5}{}{\usepackage{fontawesome5}}
\definecolor{quarto-callout-color}{HTML}{909090}
\definecolor{quarto-callout-note-color}{HTML}{0758E5}
\definecolor{quarto-callout-important-color}{HTML}{CC1914}
\definecolor{quarto-callout-warning-color}{HTML}{EB9113}
\definecolor{quarto-callout-tip-color}{HTML}{00A047}
\definecolor{quarto-callout-caution-color}{HTML}{FC5300}
\definecolor{quarto-callout-color-frame}{HTML}{acacac}
\definecolor{quarto-callout-note-color-frame}{HTML}{4582ec}
\definecolor{quarto-callout-important-color-frame}{HTML}{d9534f}
\definecolor{quarto-callout-warning-color-frame}{HTML}{f0ad4e}
\definecolor{quarto-callout-tip-color-frame}{HTML}{02b875}
\definecolor{quarto-callout-caution-color-frame}{HTML}{fd7e14}
\makeatother
\makeatletter
\makeatother
\makeatletter
\makeatother
\makeatletter
\@ifpackageloaded{caption}{}{\usepackage{caption}}
\AtBeginDocument{%
\ifdefined\contentsname
  \renewcommand*\contentsname{Table of contents}
\else
  \newcommand\contentsname{Table of contents}
\fi
\ifdefined\listfigurename
  \renewcommand*\listfigurename{List of Figures}
\else
  \newcommand\listfigurename{List of Figures}
\fi
\ifdefined\listtablename
  \renewcommand*\listtablename{List of Tables}
\else
  \newcommand\listtablename{List of Tables}
\fi
\ifdefined\figurename
  \renewcommand*\figurename{Figure}
\else
  \newcommand\figurename{Figure}
\fi
\ifdefined\tablename
  \renewcommand*\tablename{Table}
\else
  \newcommand\tablename{Table}
\fi
}
\@ifpackageloaded{float}{}{\usepackage{float}}
\floatstyle{ruled}
\@ifundefined{c@chapter}{\newfloat{codelisting}{h}{lop}}{\newfloat{codelisting}{h}{lop}[chapter]}
\floatname{codelisting}{Listing}
\newcommand*\listoflistings{\listof{codelisting}{List of Listings}}
\makeatother
\makeatletter
\@ifpackageloaded{caption}{}{\usepackage{caption}}
\@ifpackageloaded{subcaption}{}{\usepackage{subcaption}}
\makeatother
\makeatletter
\@ifpackageloaded{tcolorbox}{}{\usepackage[skins,breakable]{tcolorbox}}
\makeatother
\makeatletter
\@ifundefined{shadecolor}{\definecolor{shadecolor}{rgb}{.97, .97, .97}}
\makeatother
\makeatletter
\makeatother
\makeatletter
\makeatother
\ifLuaTeX
  \usepackage{selnolig}  % disable illegal ligatures
\fi
\IfFileExists{bookmark.sty}{\usepackage{bookmark}}{\usepackage{hyperref}}
\IfFileExists{xurl.sty}{\usepackage{xurl}}{} % add URL line breaks if available
\urlstyle{same} % disable monospaced font for URLs
\hypersetup{
  pdftitle={MTH 401 -- Real Analysis},
  colorlinks=true,
  linkcolor={blue},
  filecolor={Maroon},
  citecolor={Blue},
  urlcolor={Blue},
  pdfcreator={LaTeX via pandoc}}

\title{MTH 401 -- Real Analysis}
\usepackage{etoolbox}
\makeatletter
\providecommand{\subtitle}[1]{% add subtitle to \maketitle
  \apptocmd{\@title}{\par {\large #1 \par}}{}{}
}
\makeatother
\subtitle{Spring 2026, University of Portland}
\author{}
\date{}

\begin{document}
\maketitle
\ifdefined\Shaded\renewenvironment{Shaded}{\begin{tcolorbox}[frame hidden, sharp corners, borderline west={3pt}{0pt}{shadecolor}, interior hidden, boxrule=0pt, enhanced, breakable]}{\end{tcolorbox}}\fi

\renewcommand*\contentsname{Table of contents}
{
\hypersetup{linkcolor=}
\setcounter{tocdepth}{3}
\tableofcontents
}
\hypertarget{welcome}{%
\section*{Welcome!}\label{welcome}}
\addcontentsline{toc}{section}{Welcome!}

Before we get into the details of the course, I want to share with you a
few thoughts about my general approach to teaching and learning. My main
goal as the professor in this course is to help you succeed in not just
learning the material but in growing as a learner. Learning takes
effort, a willingness to try something that may not work, and the
ability to use feedback to refine your understanding. My job is to
foster an environment in which each and every one of you are supported
in these aspects.

Much of the work for this course will be done within class or small
group discussions for which we will rely on everyone in this class as a
source for feedback and support. Because of this, it's important that we
create an inclusive community that is respectful of our differences and
offers space for the boundary-setting necessary for positive
relationships to form. Our diversity is reflected by differences in
race, gender, sexuality, ability, class, religion, nationality, and
other cultural identities and material circumstances.

\hypertarget{general-information}{%
\section{General Information}\label{general-information}}

\begin{itemize}
\tightlist
\item
  \textbf{Instructor:} Chris Hallstrom, PhD (he / him / his).
\item
  \textbf{Office:} Buckley Center 270
\item
  \textbf{Email:}
  \href{mailto:hallstro@up.edu}{\nolinkurl{hallstro@up.edu}}.
\item
  \textbf{Webpage:} All course content will be posted on
  \url{learning.up.edu} \href{https://learning.up.edu}{Moodle}
\item
  Dr.~Hallstrom's \href{https://calendly.com/hallstro}{Calendly link}.
\end{itemize}

\hypertarget{course-description}{%
\subsection{Course Description}\label{course-description}}

In this course, we will revisit the fundamental concepts and results
from Calculus with the goal of understanding better how the basic
theorems and results follow from foundational definitions and axioms. We
will consider the question of how we can make these concepts and
arguments precise as well as how we can communicate mathematical ideas
clearly. Specific topics will include the axioms for a complete ordered
field, the formal defintion of the limit, continuity for both sequences
and functions, and the derivative. We will investigate and prove various
properties and consequences that follow from these topics.

In addition to content learning, we will also focus on the task of
communication of rigorous mathematical argument. By the end of the
semester, you should be able to

\begin{itemize}
\tightlist
\item
  effectively communicate mathematical ideas orally using correct
  terminology and language
\item
  develop and write correct and complete mathematical proofs using the
  standards and conventions of the discipline.
\end{itemize}

\hypertarget{inquiry-based-learning}{%
\subsubsection{Inquiry Based Learning}\label{inquiry-based-learning}}

We will use a method of instruction often called Inquiry Based Learning
(IBL) which is designed to engage and foster skills and habits that
working mathematicians use regularly; you will be asked to solve
problems, make conjectures, experiment, explore, create, collaborate,
and communicate your work with your peers. Rather than giving you facts
to memorize or showing you clear paths to solutions, my role is to guide
you via a sequence of carefully chosen problems through a journey of
mathematical discovery.

Throughout the semester, you will receive a list of definitions to
interpret and make sense of, as well as exercises and theorems which you
and your classmates will answer or prove. There will be very little
traditional lecture. Instead, class time will consist of student
presentations of new material. For best results, you should come to
class prepared to share your work or ideas about that day's problems.
This method of inquiry does not work nearly as well if you're looking at
problem for the first time in-class.

You will be asked to share your solutions and explain your thinking in
class regularly. You will be asked to generate examples and
counterexamples, to make conjectures, and to prove or disprove these
conjectures. When observing other student presentations, it is your
responsibility to follow their argument closely and decide if their
explanations are reasonable. If you cannot follow their logic, or have
questions about their solution, it is your responsibility to ask!

A key feature of our IBL method is student \textbf{discovery} and
therefore \textbf{outside resources are not allowed}. This means that
you should not consult any texts (other than the notes handed out in
class), the internet, students not currently enrolled in the course, or
faculty other than myself. Consulting outside resources will deprive you
of opportunities to engage with the material. You are encouraged to work
with your classmates on the problems, although every student must write
up and submit their \emph{own} solutions to written work. \textbf{You
can always ask me for help if you're stuck on a problem.}

\hypertarget{course-materials}{%
\subsection{Course Materials}\label{course-materials}}

We will use a set of course notes that I will provide. Both hard copies
and electronic versions will be available. No other materials are
needed.

\hypertarget{attendance}{%
\subsection{Attendance}\label{attendance}}

Since we will be doing much of our learning through collaborative
in-class activities, it's important that you come to class regularly,
prepared to engage and participate. Everyone's contributions are valued!

That said, I also recognize that for many reasons, this is not always be
possible. If you do need to miss class \textbf{for any reason} just let
me know -- there is no penalty apart from missing out on that day's
activity. If you miss class often, you can expect me to reach out to see
how I can support you coming to class.

I will do my best to post on Moodle a short summary of what we do in
class, so if you do miss class, you should look there to see what we did
that day. You may also find it helpful if you're able to check-in with a
classmate to see what we covered. Finally, I'm always available in
office hours to discuss anything you missed.

\href{https://www.loc.gov/programs/poetry-and-literature/poet-laureate/poet-laureate-projects/poetry-180/all-poems/item/poetry-180-013/did-i-miss-anything/}{Poem
013: Did I Miss Anything? by Tom Wyman}

\hypertarget{ai-policy}{%
\subsection{AI Policy}\label{ai-policy}}

\begin{tcolorbox}[enhanced jigsaw, bottomrule=.15mm, arc=.35mm, opacityback=0, colback=white, toprule=.15mm, breakable, leftrule=.75mm, colframe=quarto-callout-important-color-frame, left=2mm, rightrule=.15mm]
\begin{minipage}[t]{5.5mm}
\textcolor{quarto-callout-important-color}{\faExclamation}
\end{minipage}%
\begin{minipage}[t]{\textwidth - 5.5mm}

Using AI in this class fundamentally undermines our learning goals and
\textbf{should not be used in any way}.

\end{minipage}%
\end{tcolorbox}

Here's the thing -- your goal in this class is \textbf{not} to perform
calculations, to work through examples, or even to get the right answer.
Yes, we will do those things but they're all in service of the
\textbf{actual goal} which is to \textbf{think} and to
\textbf{communicate} mathematically.

Struggling to build your own understanding and to make sense of
difficult concepts is precisely the point. Using Gen AI to do these
tasks for you is to give up those opportunities to learn. Don't
undermine your growth as a human learner by looking for shortcuts!

My role is to help guide your thinking and to help you think more like
an expert and to do that I need to see and hear \emph{your} thinking. If
you use AI to do the work in this class:

\begin{itemize}
\tightlist
\item
  you haven't done your job because you haven't produced a record of
  your own thinking;
\item
  I can't do my job because I don't have access to your thinking. Worse
  than that, you are essentially asking me to read and respond to
  something that you didn't write, which is profoundly disrespectful of
  my time and energy.
\end{itemize}

\includegraphics{./images/sousanis.jpg}

Bottom line: I do not see any practical, helpful, or ethical way to use
Gen AI and LLMs to assist you in achieving this goal.

While I'm not interested in being an AI cop, you should be clear about
how this policy will be enforced

\begin{itemize}
\tightlist
\item
  If I have any concerns or questions about whether a submission
  represents your own work, I reserve the right to ask you to meet with
  me so you can explain your thinking before I give feedback.
\item
  Any submission that's not a record of your own thinking will be
  considered a violation of
  \href{https://up.smartcatalogiq.com/en/2022-2023/bulletin/university-academic-regulations/i-code-of-academic-integrity/guidelines-for-implementation-of-the-universitys-code-of-academic-integrity}{UP's
  Academic Integrity policy} and will be handled accordingly.
\end{itemize}

It goes without saying that this AI policy applies to this course and
your other courses may have different policies.

\hypertarget{additional-reading}{%
\subsubsection{Additional Reading}\label{additional-reading}}

If you would like to understand more about how I came to this position
on AI in the classroom, here are a few articles that might interest you.

\begin{itemize}
\item
  \href{https://emergingethics.substack.com/p/why-were-not-using-ai-in-this-course}{Why
  We're Not Using AI in This Course, Despite It's Obvious Benefits} by
  Patrick Lin.
\item
  \href{https://anthonymoser.github.io/writing/ai/haterdom/2025/08/26/i-am-an-ai-hater.html}{I
  Am An AI Hater} by Anthony Moser. A succinct summary of the many
  (many) ethical issues around AI tools.
\item
  \href{https://thebullshitmachines.com/}{Bullshit Machines} is a good
  explainer on how LMMs work.
\end{itemize}

\hypertarget{student-support}{%
\section{Student Support}\label{student-support}}

Research suggests that our best learning happens when we work in a zone
of \emph{productive discomfort}, meaning that we feel challenged and a
bit outside of our comfort zone. This means that as you're working
through the material - whether completing homework tasks or reviewing
class notes - \textbf{you should have questions}. While much of our
in-class time will be spent trying to address those questions, there may
not always be enough time to get all your questions answered during
class; or you might not want to wait until class to address your
questions. At the same time, it's important that you do not feel
completely stuck (at least for too long). For these reasons, it's
extremely important to me that you have resources to support your
learning outside of class.

\hypertarget{drop-in-hours}{%
\subsection{Drop-In Hours}\label{drop-in-hours}}

I will post on Moodle several blocks of time during the week that I am
available for drop-in help. You do not need to let me know you're coming
-- just stop by my office (BC 270). Many students find these drop-in
hours can be particularly helpful if you are working together with
classmates. You can work together at one of the tables down the hall
from my office and just pop in when you have any questions.

\hypertarget{sign-up-hours}{%
\subsection{Sign-Up Hours}\label{sign-up-hours}}

While I will do my best to provide multiple options for drop-in hours, I
recognize that these times might not work well with everyone's schedule.
Or you simply might find it more convenient to meet with me at a
different time. If you go to my
\href{https://calendly.com/hallstro}{Calendly scheduler}, you can
sign-up for a time slot to meet. If there is a specific time that works
for you and you don't see it available on the Calendly scheduler, please
reach out to me via email and we will find a time that works for your
schedule.

\hypertarget{open-door}{%
\subsection{Open Door}\label{open-door}}

I have the scheduled drop-in hours simply to give you some times when
you know that I'll be available -- but you are \textbf{always} welcome
to stop by my office \textbf{at any time}. Unless I'm in class or in a
meeting, I will generally be available to meet with you.

\hypertarget{course-structure}{%
\section{Course Structure}\label{course-structure}}

\hypertarget{daily-tasks}{%
\subsection{Daily Tasks}\label{daily-tasks}}

Your standing assignment in this course is to work through and prepare
complete solutions or proofs for all of the problems in the class notes.
You should come to class prepared to present and discuss your ideas,
both in small groups and as a whole class. If you get stuck on a
problem, come prepared to ask questions.

As we go, you should keep track of what problems we cover each day so
that you can anticipate what problems we will cover in the next class. I
will post on Moodle a running account of what material we cover each day
so that if you do happen to miss class, you will know how far we got in
the notes.

\hypertarget{written-homework}{%
\subsection{Written Homework}\label{written-homework}}

Roughly once per week, I will ask you to hand in write-ups of a few
selected problems. The goal of these written assignments is to
demonstrate both your understanding of the material as well as your
abilty to communicate that understanding in writing. I will provide
feedback on your work after which you are welcome to revise and resubmit
for further feedback if you wish. You may choose to use some of these
problems as evidence of your progress in the course. Guidelines for
written work can be found in \textbf{?@sec-writing}.

You are welcome to write your homework assignments by hand, but you
might also choose use this opportunity to learn to use \(\LaTeX\) which
is the typesetting system your professors use to write documents that
have lots of math notation. There are several free ways to use
\(\LaTeX\), the easiest of which is probably the web-based
\href{http://overleaf.com}{Overleaf}. Since it's web-based, there's
nothing to install. Look for the free student version. I can provide
templates to help get you started.

\hypertarget{due-dates-and-late-work}{%
\subsubsection{Due Dates and Late Work}\label{due-dates-and-late-work}}

Due dates for homework (and other assignments) are there to help both
you and me to plan our time and stay organized. They're important, but
there is usually a certain amount of flexibility and so if something
comes up that's going to make it difficult (or impossible) to complete
an assignment on time, just let me know -- preferably by sending me an
email letting me know when I can expect to get your assignment.

Here are some specific guidelines regarding extensions:

\begin{itemize}
\tightlist
\item
  Because homework is designed to help engage with material as we cover
  it in class, there's usally not much benefit from putting it off for
  too long. For this reason, I generally will not accept work that is
  more than one week late.
\item
  There is no penalty for late work \textbf{except} that you will not
  get timely feedback from me on your work. This could impact your
  ability to include problems in your portfolio (see below).
\item
  If your work is consistently late, you can expect that I will reach
  out to see if we can work together to find ways for you to keep up
  with the work in the course.
\item
  To prevent work from backing up, there are two \textbf{hard} deadlines
  in this course - one is February 27th (the Friday before spring break)
  and the other is April 24th (the last day of classes). Late work will
  not be accepeted after those dates, except by prior arrangment.
\end{itemize}

\hypertarget{quizzes}{%
\subsection{Quizzes}\label{quizzes}}

Throughout the semester, we may have a number of quizzes to help us
assess your understanding of the material. While these won't be graded
in the traditional sense, I will provide feedback which you can use to
help your learning. You may also choose to include this work in your
portfolio.

\hypertarget{proof-portfolio}{%
\subsection{Proof Portfolio}\label{proof-portfolio}}

As we proceed through the semester, I will ask you to collect samples of
your work that exemplify your understanding and engagement with the
class content. At mid-semester, I will ask to see a preliminary draft to
help ensure you're making progress. The final version is due on the
Friday of the last week of classes.

\begin{itemize}
\tightlist
\item
  Mid-semester check-in due Feb.~27th.
\item
  Final portfolio due April 24th.
\end{itemize}

Your portfolio should include problems from class presentations or
weekly homework that provide evidence of your understanding of the
course material as well as your learning over the semester. I'm
particularly interested in seeing how you have developed in your
learning or have responded to feedback.

You should include a short reflection for each work element that you
include, describing why you chose that particular solution/proof and how
you see it demonstrating your learning.

\hypertarget{finals-week}{%
\subsection{Finals Week}\label{finals-week}}

We not have a traditional final exam, but we will use the scheduled exam
time, \textbf{Tuesday 1:30-3:30}, for short one-on-one conferences. This
will give me a chance to ask you any questions I might have about your
portfolio, as well as for to tell me anything else about your learning
that isn't captured in your portfolio.

Additionally, there will be a short final reflection paper to be
completed during finals week which will give you an opportunity to
reflect on and synthesize some of the main themes of the course. As part
of this, you will also be asked to self-assess your learning over the
course of the semester.

\hypertarget{grades}{%
\section{Grades}\label{grades}}

\begin{quote}
Extrinsic motivation, which includes a desire to get better grades, is
not only different from, but often undermines, intrinsic motivation, a
desire to learn for its own sake.

-- Alfie Kohn,
\href{https://www.alfiekohn.org/article/case-grades/}{``The Case Against
Grades''}
\end{quote}

Grades, as they are traditionally thought of, are inherently imprecise
and don't represent a full picture of your growth and learning over the
course of a semester. Worse than that, research suggests that grades
undermine the learning process in several ways:

\begin{itemize}
\tightlist
\item
  Grades tend to diminish interest in what you're learning.
\item
  Grades create a preference for the easiest task. In other words,
  students tend to do what they need to get a certain grade, but no
  more.
\item
  Grades tend to reduce the quality of student thinking. The moment we
  ask ``\textbf{how} am I doing?'' we lose track of \textbf{what} we're
  doing.
\end{itemize}

Although I am required to submit a grade for each student at the end of
the semester, I will do what I can to de-emphasize the role of grades so
that as much as possible our focus is on learning.

\hypertarget{collaborative-grading}{%
\subsection*{Collaborative Grading}\label{collaborative-grading}}
\addcontentsline{toc}{subsection}{Collaborative Grading}

Rather than giving you marks on individual assignments, I will instead
give you extensive feedback on your work. After addressing that
feedback, you're welcome to resubmit for further feedback if you wish.
Throughout the semester, I will periodically ask you to reflect
carefully on your work and to evaluate your progress. You will collect
evidence of your understanding of the course content and based on that
evidence, you will be asked to suggest a final course grade. In this
way, we will determine your your grade collaboratively.

The intention here is to help you focus on learning in a way that is
more organic, as opposed to simply working as you think you're expected
to. If this process causes more anxiety than it alleviates, please see
me at any point to confer about your progress in the course -- I'm
always happy to talk with you about your learning!

Here some of the ways that you can demonstrate your understanding of the
course material:

\begin{itemize}
\tightlist
\item
  Present correct solutions / valid proofs in class.
\item
  Submit weekly homework (including revisions that incorporate my
  feedback) that reflects understanding of specific topics.
\item
  When presenting in class, respond thoughtfully to questions.
\item
  When listening to presentations, ask thoughtful questions.
\item
  Provide complete solutions on quizzes.
\item
  Submit a proof portfolio that demonstrates your understanding through:

  \begin{itemize}
  \tightlist
  \item
    Finding and demonstrating connections between ideas.
  \item
    Constructing examples and non-examples that demonstrate
    understanding of definitions.
  \item
    Correctly using and explaining the role of axioms and definitions.
  \end{itemize}
\end{itemize}

\hypertarget{qualitative-descriptions-of-grades}{%
\subsection*{Qualitative Descriptions of
Grades}\label{qualitative-descriptions-of-grades}}
\addcontentsline{toc}{subsection}{Qualitative Descriptions of Grades}

In thinking about grades, I find it helpful to begin with qualitative
descriptions of what a particular grade might represent in terms of
learning.

\begin{description}
\item[A]
This grade generally indicates superior work that demonstrates a deep
and thorough understanding of all material such that you could likely
apply your understanding to unfamiliar or especially complex problems.
Written work is clear, easy to read, and logically correct. You
consistently demonstrate your deep understanding using most of the
methods described above, including a portfolio.
\item[B]
This grade indicates good work that is eminently satisfactory. You have
a solid understanding of most of the topics we've coverd, although there
may be a few gaps. You would likly be able apply your understanding to
some new situations although you might have difficulty with particularly
challenging or unfamilar problems. You have demonstrated your
understanding using most of the methods described above, including a
portfolio.
\item[C]
This grade indicates competent work that demonstrates a basic
understanding of course topics, although significant gaps remain. You
can handle straightforward problems similar to ones seen throughout the
semester, but would likely struggle with novel or more challenging
problems. You have demonstrated your understanding using some of the
methods described above.
\item[D/F]
These grades represent a fundamental breakdown of expectations. A D
represents a meaningful but unsuccessful attempt at earning a C or
above. An F represents such a severe lack of engagement, effort, or
understanding that there is no evidence of meaningful progress.
\end{description}

\hypertarget{engagement}{%
\subsection*{Engagement}\label{engagement}}
\addcontentsline{toc}{subsection}{Engagement}

Although your course grade should be based on your
\textbf{understanding} of course content and not on course engagement,
in my experience these typically go hand in hand. So while engagement in
the course is not itself evidence of understanding, it does usually help
us achieve that goal.

Here are some ways that you can engage with the class:

\begin{itemize}
\tightlist
\item
  Attend class regularly
\item
  Work ahead on new problems and come to class prepared to discuss
\item
  Work to make sense of new definitions or axioms.
\item
  Ask questions - either in class or in drop-in hours.
\item
  Volunteer to present your work. If you're not comfortable sharing your
  work with the class, you can always share with me during drop-in
  hours.
\item
  Actively participate in discussions and group work. This can mean
  sharing your work but it can also mean asking questions or helping to
  facilitate the discussion.
\item
  Support your classmates and help them succeed.
\end{itemize}

\hypertarget{partial-grades}{%
\subsection*{Partial Grades}\label{partial-grades}}
\addcontentsline{toc}{subsection}{Partial Grades}

In discussing your course grade together, we may opt to add a modifier
to your grade to acccount for factors such as your level of engagement
with the course.

\hypertarget{university-policies-resources}{%
\section{University Policies \&
Resources}\label{university-policies-resources}}

\hypertarget{code-of-academic-integrity}{%
\subsection{Code of Academic
Integrity}\label{code-of-academic-integrity}}

The University of Portland is a diverse academic community of learners
and scholars who are dedicated to freely sharing ideas and engaging in
respectful discussion of those ideas to discover truth. Such pursuits
require each person, whether student or faculty, to present truthfully
our own ideas and give credit to others for the ideas that they
generate. Thus, cheating on exams, copying another student's assignment,
including homework, or using the work of others without proper citation
are some examples of violating academic integrity.

Especially for written and oral assignments, students have an ethical
responsibility to properly cite the authors of any books, articles, or
other sources that they use. Students should expect to submit
assignments to Turnitin, a database that ensures assignments are
original work of the student submitting. Each discipline has guidelines
for how to give appropriate credit, and instructors will communicate the
specific guidelines for their discipline. The Clark Library also
maintains a webpage that provides citation guidelines at
\href{https://libguides.up.edu/cite}{libguides.up.edu/cite}.

The misuse of AI to shortcut course learning outcomes will be treated as
a violation of academic integrity comparable to plagiarism or cheating.
Faculty are responsible for including a written ``Course AI Policy'' in
their syllabi that clearly states what they consider appropriate and
inappropriate uses of AI in the context of their courses. Students are
responsible for using AI in ways that do not detract from the
established learning outcomes of the course. All members of the
scholarly community are responsible for demonstrating sound judgment in
discerning when and how to utilize AI in their work, upholding standards
of citation, originality, and integrity.

For more information, please see the
\href{http://up.smartcatalogiq.com/en/2022-2023/bulletin/University-Academic-Regulations/I-Code-of-Academic-Integrity/Guidelines-for-Implementation-of-the-Universitys-Code-of-Academic-Integrity}{Guidelines
for Implementation of the University's Code of Academic Integrity} in
the University Bulletin.

\hypertarget{assessment-disclosure}{%
\subsection{Assessment Disclosure}\label{assessment-disclosure}}

Student work products for this course may be used by the University for
educational quality assurance purposes. For reasons of confidentiality,
such examples will not include student names.

\hypertarget{accessibility}{%
\subsection{Accessibility}\label{accessibility}}

The University of Portland strives to make its courses and services
fully accessible to all students. Students are encouraged to discuss
with their instructors what might be most helpful in enabling them to
meet the learning goals of the course. Students who experience a
disability are also encouraged to use the services of the Office for
Accessible Education Services (AES), located in the Shepard Academic
Resource Center (503-943-8985).

\textbf{If you have an AES Accommodation Plan}, you should meet with
your instructor to discuss how to implement your plan in this class.
Requests for alternate location for exams and/or extended exam time
should, where possible, be made two weeks in advance of an exam, and
must be made at least one week in advance of an exam. Also, if
applicable, you should meet with your instructor to discuss emergency
medical information or how best to ensure your safe evacuation from the
building in case of fire or other emergency. All information that
students provide regarding disability or accommodation is confidential.
All students are responsible for completing the required coursework and
are held to the same evaluation standards specified in the course
syllabus.

\hypertarget{mental-health}{%
\subsection{Mental Health}\label{mental-health}}

Anyone can experience problems with their mental health that interfere
with academic experiences and negatively impact daily life. If you or
someone you know experiences mental health challenges at UP, please
contact the \href{https://www.up.edu/counseling/}{University of Portland
Counseling Center} in the upper level of Orrico Hall (down the hill from
Franz Hall and near Mehling Hall) at 503-943-7134 or
\href{mailto:hcc@up.edu}{\nolinkurl{hcc@up.edu}}. Their services are
free and confidential. In addition, mental health consultation and
support is available through the Pilot Helpline by calling 503-943-7134
and pressing 3. The University of Portland Campus Safety Department
(503-943-4444) also has personnel trained to respond sensitively to
mental health emergencies at all hours. Remember that getting help is a
smart and courageous thing to do -- for yourself, for those you care
about, and for those who care about you. For more information on health
and wellness resources at UP go to
\href{https://www.linktr.ee/wellnessUP}{www.linktr.ee/wellnessUP}.

\hypertarget{non-violence}{%
\subsection{Non-Violence}\label{non-violence}}

The University of Portland is committed to fostering a safe and
respectful community free from all forms of violence. Violence of any
kind, and in particular acts of power- based personal violence, are
inconsistent with our mission. Together, all UP community members must
take a stand against violence. Learn more about what interpersonal
violence looks like, campus and community resources, UP's prevention
strategy, and what we as individuals can do to assist on the
\href{https://www.up.edu/greendot}{Green Dot website}. Further
information and reporting options may be found on the
\href{htts://www.up.edu/titleix}{Title IX website}.

\hypertarget{ethics-of-information}{%
\subsection{Ethics of Information}\label{ethics-of-information}}

The University of Portland is a community dedicated to the investigation
and discovery of processes for thinking ethically and encouraging the
development of ethical reasoning in the formation of the whole person.
Using information ethically, as an element in open and honest scholarly
endeavors, involves moral reasoning to determine the right way to
access, create, distribute, and employ information, including:
considerations of intellectual property rights, fair use, information
bias, censorship, and privacy. More information can be found in the
Clark Library's guide to the
\href{https://libguides.up.edu/ethicaluse}{Ethical Use of Information}.

\hypertarget{final-exam-policy}{%
\subsection{Final Exam Policy}\label{final-exam-policy}}

The University's Academic Regulation regarding final examinations states
that these may only be given during the scheduled times published by the
registrar {[}noted above for the course{]}. During the week prior to
final examination week, no examinations may be given, except in
laboratory practica. Regardless of whether a final exam is given, all
classes must meet during final examination week in accordance with the
final exam schedule.

\hypertarget{the-learning-commons}{%
\subsection{The Learning Commons}\label{the-learning-commons}}

Students may receive academic assistance through Learning Commons
tutoring services and workshops. The Co-Pilot peer tutoring program
provides students with opportunities to work with other students to get
help in writing, math, group projects, and many other courses. Schedule
an appointment to meet with a Co-Pilot (tutor) by visiting the
\href{https://www.up.edu/learningcommons}{Learning Commons website}.
Students can also meet with a Co-Pilot during drop-in hours. Check the
Learning Commons website or stop by the Learning Commons in BC 163 to
learn more about their services. Co-Pilots are a wonderful support along
your academic journey.

\hypertarget{withdrawal-procedures}{%
\subsection{Withdrawal Procedures}\label{withdrawal-procedures}}

It is the student's responsibility to drop the course if he or she is no
longer planning on attending the course or filling the other course
requirements. In order to drop, the student must use and Add/Drop form
available at the Registration Office. If a student does not properly
withdraw from a course, he or she may receive an \textbf{F} for the
course. A properly withdrawn student will receive a \textbf{W}. The last
day to withdraw is \textbf{Monday, April 13.}

\hypertarget{incompletes}{%
\subsection{Incompletes}\label{incompletes}}

An incomplete (\textbf{I}) will only be considered when the quality of a
student's work is satisfactory (C- or better), but for some essential
reason the course has not been completed by the student. An (I) is
reserved for emergency situations only. To request an incomplete, the
student must submit a typed, signed and dated letter stating the
reason(s) why an incomplete is appropriate. The letter should also
contain the conditions for the completion of work. Acceptance of the
request shall be at the discretion of the instructor, Department Chair,
and/or Dean of the College of Arts \& Sciences.



\end{document}
